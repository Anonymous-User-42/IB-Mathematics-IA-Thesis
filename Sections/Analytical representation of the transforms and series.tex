

\section{\textit{Analytical Representation}}

	\textit{When analyzing and investigating the Laplace transform, which was previously defined as,}
	
		$$\mathcal{L}\{f\}(s) = F(s) = \int_0^{\infty}f(t)e^{-\sigma t}e^{-iwt} dt$$	
	
	\textit{When $\sigma = 0$, the transform reduces to,}	
	
		$$\mathcal{L}\{f\}(s) = F(s) = \int_0^{\infty}f(t)e^{-iwt} dt = \left.\mathcal{F}\{f\}(s)\right|_{0}^{\infty}$$

	\textit{Therefore, the Fourier transform is the Laplace transform when the complex variable, s has no real part. ie. $\sigma = 0 \implies s = iw$.}

	\textit{Visually, the Fourier transform is the Laplace transform at $\sigma = 0$. It is a slice of the Laplace transform over the whole domain.}



