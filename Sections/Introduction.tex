%\textit{I have chosen this research question because the fundamental relation between the \textbf{Laplace transform} and its effects on the the \textbf{applied function} and its excruciating difficulty in possibly modeling this behavior has led me to explore this field, as this mathematical concept is exciting, unique and wonderful, with ample scope to study and analyze.}


%{I have chosen this research question because I intend to study \textbf{Electrical Engineering}. The fundamental concepts of the \textbf{Laplace transform}, \textbf{Fourier transform} and the \textbf{Fourier series} form the \textbf{backbone} of a subfield in Electrical Engineering, that is "\textbf{Signal processing}", without which any sort of \textbf{communication}, through any \textbf{electrical device} would not be possible, which implies that \textbf{modern day computers} and \textbf{computational technology} that exists today would also \textbf{not be existent}.}

%{The very idea that all of computational technology depends on the concepts laid out \textbf{150} to \textbf{200 years ago}, deeply interests me, and thus has led me to \textbf{study the complex relationships} within \textbf{various mathematical concepts}.}

{The brachistochrone problem, first proposed by Galileo and rediscovered by Johann Bernoulli in 1697, is one of mathematics' most intriguing solved problems. The brachistochrone curve is named after the basic words brachistos (shortest) and chrone (time). This problem is lovely not only because of the question's simplicity but also because of the numerous solutions it encourages. We may observe some of the finest brains in mathematics wrestle and struggle to create more information for all through this puzzle.}

{Simply put, the reader is asked to find a line connecting two points. According to Euclid's first postulate, a straight line segment can always be drawn connecting any two points. Between two locations on a Euclidian surface, the shortest path is automatically this line segment. What if we didn't want to determine the fastest path between these two sites but rather the shortest time?}

{Assume a string has a bead threaded on it, and the bead can freely move from point A to point B due to the absence of friction and drag forces. What curve should the string be in this situation, with a constant downward acceleration \textit{g}, to reduce the bead's journey time?}

{This question may appear to the reader as a simple minimization problem at first glance. All calculus students well understand the potency of calculus in this aspect. When a function needs to be reduced, the derivative of that function equated to zero indicates the minimum and maximum points of the function.}

{Using this logic, we are to find a function that minimizes the travel time from point A to point B.}

{I have chosen this topic for my exploration as I am a physics student, physics fanatic and a deep admirer of the beauty of calculus. I intend to study and explore the mathematics behind this simple mechanics phenomenon from an high-school student perspective. This topic is the due reason for the major advancements in calculus, by Bernoulli, Euler and Lagrange.}

{The very idea that a very simple mechanics problem that is reason behind great interesting developments in mathematics has caused me to think and explore this problem and the mathematics underlying this problem using the mathematics I have learnt during my mathematics AA course and using some undergraduate mathematics that I have learnt in my free time as a matter of pure interest.}


