%\textit{I have chosen this research question because the fundamental relation between the \textbf{Laplace transform} and its effects on the the \textbf{applied function} and its excruciating difficulty in possibly modeling this behavior has led me to explore this field, as this mathematical concept is exciting, unique and wonderful, with ample scope to study and analyze.}


%{I have chosen this research question because I intend to study \textbf{Electrical Engineering}. The fundamental concepts of the \textbf{Laplace transform}, \textbf{Fourier transform} and the \textbf{Fourier series} form the \textbf{backbone} of a subfield in Electrical Engineering, that is "\textbf{Signal processing}", without which any sort of \textbf{communication}, through any \textbf{electrical device} would not be possible, which implies that \textbf{modern day computers} and \textbf{computational technology} that exists today would also \textbf{not be existent}.}

%{The very idea that all of computational technology depends on the concepts laid out \textbf{150} to \textbf{200 years ago}, deeply interests me, and thus has led me to \textbf{study the complex relationships} within \textbf{various mathematical concepts}.}

{The brachistochrone problem is a mathematical problem/challenge proposed by Bernoulli to fellow mathematicians in 1696. The brachistochrone curve is named after the simple words in Latin, brachistos (shortest) and chrone (time).}

{While exploring and researching for my Mathematics IA topic, I came across this topic, which at first sounded like a simple minimization problem whose solution can be found with high-school calculus. But after further reading and understanding the problem, I realized that the problem is quite complicated, though it seems simple.}

{The problem is about finding a curve between two points (where one point is higher than the other) on a two dimensional plane, such that a frictionless object travels in the shortest time possible. One might think it is a straight line, as I had initially thought, but this is the path for the shortest distance, not the shortest time.}

{I visualized this problem in such a way that if a bead is to be threaded on a string that can freely move between two points, namely A and B in the absence of drag and frictional forces}

{This question may appear to the person as a simple minimization problem at first glance. All calculus students well understand the potency of calculus in this aspect. When a function needs to be reduced, the derivative of that function equated to zero indicates the minimum and maximum points of the function.}

{Using this logic, we are to find a function that minimizes the travel time from point A to point B.}

{I aim to explore this problem from a high-school student perspective using high-school mathematics that I have learnt during my mathematics AA course and using some undergraduate mathematics that I have learnt in my free time.}

{The very idea that a straightforward mechanics problem is a reason behind great interesting developments in mathematics has caused me to explore this problem.}

{To solve this problem, I have to find the curve of fastest descent and to find that, I would need to find a function that optimizes time. This would be the first step to solve the problem.}


