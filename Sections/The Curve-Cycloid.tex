

{In the process of using the Calculus of Variations, to minimize the time integral, I have come across the two defining parametric equations of a cycloid.}

{A cycloid is the curve formed when traced by a point on a circle as it rolls along a straight line without it slipping.}

{I know this for a fact, as the standard cycloid equations are of the form, $x = r\cos\theta$ and $y = r\sin\theta$.}

{But if we are to allow the circle that forms the cycloid curve to rotate in a clockwise direction with an angle $\theta$ from the bottom of the circle, the above standard equations representing the cyloid curve must be corrected as follows,}

	$$x = -r\sin\theta, y = r\cos\theta$$

{Also as the circle that forms the cyloid curve is to move in the positive \textit{x} direction, I have to embed this periodic movement mathematically in the parametric functions of the cycloid. Therefore we have,}

	$$x = -r\sin\theta + \Delta x$$

	$$\Delta x = 2\pi r\cdot\frac{\theta}{2\pi} = r\theta$$

{Therefore we have,}

	$$x = -r\sin\theta + r\theta = r\left(\theta - \sin\theta\right)$$

{However in the \textit{y} direction, the only correction that needs to be done is the generalization, that the center of the circle that forms the cycloid is at $(r,r)$ and not at $(0,0)$, so as to ensure that t he bottom of the cycloid rest at the \textit{x} axis. Therefore with a vertical translation of \textit{r} units in the \textit{y} axis we have,}

	$$y = r - r\cos\theta = r\left(1 - \cos\theta\right)$$

{Therefore we have,}

	\begin{equation}
		\boxed{\begin{array}{rc1}x = r\left(\theta - \sin\theta\right) \\ y = r\left(1 - \cos\theta\right) \\ \end{array}}
		\label{cyceq}
	\end{equation}

{From the derivation above, I have confirmed that the parametric equations, that are a solution to  the brachistorcone problem are indeed, that of a cycloid, as both of the curves have the same form.}

{On comparison with the parametric equations I have found as a solution to the brachistochrone problem with the improvised parametric equations of a cycloid, we observe that, $r = k_{1}/2$. In other words, $k_{1}$ is equal to the diameter of the circle which forms the cycloid}


