

\section{\textit{Types of Ordinary Differential Equations}}
	
	\textit{There are various types of Ordinary Differential Equations based on the specific values/functions that replace the the values of the coefficients of the differential equation, order and degree of the equation.}	
	
	\subsection{\textit{Linear Ordinary Differential Equations}}

		\textit{Linear Ordinary Differential Equations are differential equations where, the degree/exponent of the differential equation is 1 and the coefficients of the differential equation are some arbitrary differential functions in the base variable.}

	\subsection{\textit{Non-Linear Ordinary Differential Equations}}

		\textit{An Ordinary Differential equation is said to be non-linear when the degree/exponent of the differential equation is anything but 1 and the co-efficient of the the differential equation is a non-differentiable function of in the principal function, ie. $\sin{y}$ as $a_k$ or $\left(\overset{k}{\dot{y}}\right)^2$ for some value k.}

	\subsection{\textit{Homogeneous Ordinary Differential Equations}}

		\textit{An Ordinary Differential equation is said to be homogeneous, when with the nth derivative of the principal function, all other derivatives of lower order's are sequentially present, ie. $a_{1}\bigg(\dddot{y}\bigg)^{k_1} + a_{2}\bigg(\ddot{y}\bigg)^{k_2} + a_{3}\bigg(\dot{y}\bigg)^{k_3} + a_{4}\bigg(y\bigg)^{k_4} = c$}

	\subsection{\textit{Heterogeneous Ordinary Differential Equations}}

	\textit{An Ordinary Differential equation is said to be non-homogeneous, when with the nth derivative of the principal function, other derivatives of lower order's are not present, ie. $a_{1}\bigg(\dddot{y}\bigg)^{k_1} + a_{2}\bigg(\dot{y}\bigg)^{k_2} + a_{3}\bigg(y\bigg)^{k_3} = c$}


