

\section{\textit{Types of Domains}}
	
	\subsection{\textit{Time Domain (Real Domain)}}
	
		\textit{The time domain, or the real domain is the domain in which we classically deal with functions, usually the x plane. In practical applications, every physical quantity is measured relatively against time, therefore the name as, the time domain.}	
	
	\subsection{\textit{Frequency Domain (Complex Domain)}}

		\textit{The frequency domain, or the complex domain is the domain which is usually extended mathematically from the real domain, with the aid of imaginary numbers and complex analysis, usually dealt in the z plane.}	

		\textit{In practical applications when a function or otherwise known as a signal (in Engineering terminologies) is transformed mathematically, it aids the observer to study with the concept of super-positioning to understand how much of each signal as a part summed up with other parts make up the original function.}

		\textit{In terms of physical quantities, this is done by studying the frequencies of each counterpart of a signal, thus its name as, the frequency domain.}


