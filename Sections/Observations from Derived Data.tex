\textit{Upon close observation of the \textbf{drag versus time} it is evident that, the drag periodically decreases with decrease in mass and increases in amplitude and frequency with steady increase in mass}

\textit{As mentioned before, the viscous drag that is in play, exhibits underdamping that is the damping ratio involved in this example is less than 1 ($\zeta < 1$).}

\textit{The system exhibits an interesting feature, that of constant logarithmic decrements, that is,}
        
    $$\ln{\frac{x_1}{x_2}} = \ln{\frac{x_2}{x_3}} = \ln{\frac{x_3}{x_4}} = \cdots\cdots\cdots$$
        
\textit{Where $x_n$ and $x_{n + 1}$ are the amplitudes of any two successive peaks ($n \in \mathbb{R}$).}
        
\textit{Also interestingly that for any two successive peaks, if we define,}
        
    $$\delta = \ln{\left(\frac{x_n}{x_{n + 1}}\right)}$$
        


