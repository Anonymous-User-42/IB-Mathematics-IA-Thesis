\newpage

%\chapter{\textit{Python Source Codes}}\label{pycode}
        
%    \section{\textit{Graphical Solution Codes}}

	\section{Appendix: Python Source Code}
        
        \begin{lstlisting}[language=Python, caption=\textit{Python Script for the Brachistochrone system}]
import numpy as np
from scipy.optimize import newton
from scipy.integrate import quad
import matplotlib.pyplot as plt

# Acceleration due to gravity (m.s-2); final position of bead (m).
g = 9.81
x2, y2 = 4, 3

def cycloid(x2, y2, N=100):
    """Return the path of Brachistochrone curve from (0,0) to (x2, y2).

    The Brachistochrone curve is the path down which a bead will fall without
    friction between two points in the least time (an arc of a cycloid).
    It is returned as an array of N values of (x,y) between (0,0) and (x2,y2).

    """

    # First find theta2 from (x2, y2) numerically (by Newton-Rapheson).
    def f(theta):
        return y2/x2 - (1-np.cos(theta))/(theta-np.sin(theta))
    theta2 = newton(f, np.pi/2)

    # The radius of the circle generating the cycloid.
    R = y2 / (1 - np.cos(theta2))

    theta = np.linspace(0, theta2, N)
    x = R * (theta - np.sin(theta))
    y = R * (1 - np.cos(theta))

    # The time of travel
    T = theta2 * np.sqrt(R / g)
    print('T(cycloid) = {:.3f}'.format(T))
    return x, y, T

def linear(x2, y2, N=100):
    """Return the path of a straight line from (0,0) to (x2, y2)."""

    m = y2 / x2
    x = np.linspace(0, x2, N)
    y = m*x

    # The time of travel
    T = np.sqrt(2*(1+m**2)/g/m * x2)
    print('T(linear) = {:.3f}'.format(T))
    return x, y, T

def func(x, f, fp):
    """The integrand of the time integral to be minimized for a path f(x)."""

    return np.sqrt((1+fp(x)**2) / (2 * g * f(x)))

def circle(x2, y2, N=100):
    """Return the path of a circular arc between (0,0) to (x2, y2).

    The circle used is the one with a vertical tangent at (0,0).

    """

    # Circle radius
    r = (x2**2 + y2**2)/2/x2

    def f(x):
        return np.sqrt(2*r*x - x**2)
    def fp(x):
        return (r-x)/f(x)

    x = np.linspace(0, x2, N)
    y = f(x)

    # Calcualte the time of travel by numerical integration.
    T = quad(func, 0, x2, args=(f, fp))[0]
    print('T(circle) = {:.3f}'.format(T))
    return x, y, T

def parabola(x2, y2, N=100):
    """Return the path of a parabolic arc between (0,0) to (x2, y2).

    The parabola used is the one with a vertical tangent at (0,0).

    """

    c = (y2/x2)**2

    def f(x):
        return np.sqrt(c*x)
    def fp(x):
        return c/2/f(x)

    x = np.linspace(0, x2, N)
    y = f(x)

    # Calcualte the time of travel by numerical integration.
    T = quad(func, 0, x2, args=(f, fp))[0]
    print('T(parabola) = {:.3f}'.format(T))
    return x, y, T

# Plot a figure comparing the four paths.
fig, ax = plt.subplots()

for curve in ('cycloid', 'circle', 'parabola', 'linear'):
    x, y, T = globals()[curve](x2, y2)
    ax.plot(x, y, lw=4, alpha=0.5, label='{}: {:.3f} s'.format(curve, T))
ax.legend()

ax.set_xlabel('$x$')
ax.set_xlabel('$y$')
ax.set_xlim(-0.5, 4.5)
ax.set_ylim(3.5, -0.5)
plt.savefig('brachistochrone.png')
plt.show()

\end{lstlisting}
        
